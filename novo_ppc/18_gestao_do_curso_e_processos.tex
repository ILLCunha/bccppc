\chapter{GESTÃO DO CURSO E PROCESSOS DE AVALIAÇÃO INTERNA E EXTERNA}

A avaliação não está dissociada do planejamento, tanto em nível do ensino quanto em nível do curso. A avaliação configura-se como um instrumento indispensável para pensar, executar e reelaborar o planejamento. Nesse sentido, como observa Luckesi (2002, p. 93), ela exige uma decisão do que fazer com o resultado, direcionando o objeto da avaliação “numa trilha dinâmica de ação”. A prática da autoavaliação cria oportunidades para a ampliação de conhecimento, reflexão crítica e construção coletiva de diretrizes necessárias para a tomada de decisões.

Sendo um processo permanente e sistemático, a autoavaliação do curso será balizada por um \textit{projeto de autoavaliação}, cuja elaboração tenha a contribuição de gestores, docentes, discentes e técnico-administrativos. O projeto deverá conter os objetivos, metodologias, formas de divulgação e discussão dos resultados, bem como um cronograma.

A autoavaliação será diagnóstica e propositiva, apontando potencialidades e fragilidades presentes no desenvolvimento do curso em seus mais variados aspectos, tais como o rendimento acadêmico dos alunos, práticas de ensino, projetos interdisciplinares, indissociabilidade entre ensino, pesquisa e extensão, monitoria, gestão do curso, matriz curricular e conteúdo, estágios, atividades complementares, infraestrutura, alinhamento com o PPI, etc. A autoavaliação apresenta um caráter contínuo e cíclico, podendo se dar com periodicidade semestral e anual em função dos aspectos a serem avaliados.

Deve-se observar que conforme a Resolução CEPE/UFRPE nº 065/2011, com base na Resolução CONAES/MEC nº 01/2010, cabe ao NDE, como órgão consultivo, a responsabilidade pela concepção do projeto pedagógico do curso, bem como sua atualização e revitalização. O NDE também tem por atribuição a supervisão do processo de avaliação e acompanhamento do curso definidas pelo Colegiado do mesmo.

A Coordenação do Curso, por sua vez, seguirá o princípio da gestão democrática fomentando a participação dos professores, técnicos e estudantes nos processos de avaliação e planejamento. Para a consecução das ações necessárias ao desenvolvimento do curso, a Coordenação contará com assessoria técnica e apoio institucional da PREG, PROPLAN, CPA – Comissão Própria de Avaliação e NACES, além de outros órgãos da Universidade que julgar necessários.

Na análise dos resultados e consequente proposição de ações resultantes de seu processo de autoavaliação, o curso deverá atentar para o perfil do egresso, as Diretrizes Curriculares Nacionais para o Bacharelado em Ciência da Computação, os objetivos definidos neste PPC, as políticas institucionais expressas no PDI, em especial no PPI (ver item 17), e as demais avaliações realizadas no âmbito do Sistema Nacional de Avaliação da Educação Superior – SINAES (BRASIL, 2004).

O SINAES é constituído por três modalidades avaliativas: Avaliação das Instituições de Ensino Superior - AVALIES, Avaliação dos Cursos de Graduação - ACG, e ENADE. Cada uma delas é desenvolvida em situações e momentos distintos, mas devem promover articulações entre si. No caso da AVALIES, esta é composta pela avaliação institucional externa e interna. A avaliação institucional externa é realizada por comissões avaliadoras do INEP, ao passo que a avaliação institucional interna fica a cargo da Comissão Própria de Avaliação - CPA de cada instituição.

A UFRPE constituiu a sua CPA por meio da Portaria nº 313/2004-GR, de 14 de junho de 2004, com o objetivo de elaborar e desenvolver, juntamente à comunidade acadêmica, Administração Superior e Conselhos Superiores, uma proposta de autoavaliação institucional, coordenando e articulando os processos internos de avaliação da UFRPE, de acordo com princípios e diretrizes do SINAES.

Em 2015, a referida Comissão criou o \textit{Boletim CPA}, uma publicação reunindo os resultados da autoavaliação institucional da UFRPE relativos ao Eixo 03 (Políticas Acadêmicas)1. O Boletim CPA possui duas particularidades que o distinguem dos \textit{Relatórios de Autoavaliação Institucional}, enviados anualmente ao MEC, já que apresenta, especificamente, os resultados da avaliação feita pelos discentes e é organizado por curso de graduação. Ou seja, ao contrário do Relatório, de caráter abrangente, o Boletim é mais específico, trazendo as avaliações do corpo discente de cada curso sobre o ensino, a pesquisa, a extensão, a comunicação com a sociedade, e a política de atendimento aos estudantes.

A 1ª edição do Boletim CPA-UFRPE foi elaborada em 2015 com base no Questionário CPA 2014. Foram publicados quatro volumes, contemplando o campus Dois Irmãos e as três Unidades Acadêmicas existentes na época de aplicação do questionário em 2014.1, UAG, UAST e UAEADTec. Com a 2ª edição do Boletim, em 2016, foi acrescido o volume correspondente à UACSA. O principal objetivo do Boletim é auxiliar a Coordenação do curso, o NDE, juntamente com discentes, docentes e técnicos nos processos de avaliação e aprimoramento do curso. Neste sentido, a Coordenação ou o NDE poderá solicitar à CPA a realização de Encontros de Autoavaliação com uma síntese dos resultados do Boletim, de modo a discutir aspectos da autoavaliação institucional no âmbito do curso e possíveis encaminhamentos.