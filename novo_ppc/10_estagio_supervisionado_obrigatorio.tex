\chapter{ESTÁGIO SUPERVISIONADO OBRIGATÓRIO}

De acordo com a Lei nº 11.788/2008, o estágio é um ``ato educativo escolar supervisionado, desenvolvido no ambiente de trabalho'' que tem o propósito de garantir o “aprendizado de competências próprias da atividade profissional e a contextualização curricular, objetivando o desenvolvimento do educando para a vida cidadã e para o trabalho”. O Estágio Supervisionado Obrigatório (ESO), fazendo parte da matriz curricular, constitui-se num espaço de aprendizagem concreta de vivência prática. O objetivo central se direciona na aplicação dos conhecimentos científicos adquiridos durante a realização do curso e a vivência profissional.

Os mecanismos de acompanhamento e de cumprimento serão estabelecidos e acompanhados pelo Coordenador do Curso em conjunto com a Comissão de Estágio Supervisionado (COE), regidos pela resolução de ESO de BCC/UAG. As atividades referentes ao estágio poderão ser encontradas na própria resolução, documento aprovado pelo Colegiado de Coordenação Didática do Curso de Bacharelado em Ciência da Computação (CCD/BCC), contendo o detalhamento das atividades permitidas, inclusive a possibilidade de equiparação de atividades para ESO, mediante aprovação do CCD/BCC, tendo em vista que a região ainda não possui um número significativo de empresas na área de Tecnologia da Informação (TI), que possam ofertar vagas e suporte adequado aos nossos alunos. Convém ressaltar que estágios relevantes para os futuros egressos deste curso envolve atividades específicas da área, devendo seguir um processo bem definido e institucionalizado. Estes, resumidamente, consistem sistematicamente nas seguintes etapas:

\begin{enumerate}
    \item Matrícula na disciplina de ESO;
    \item Solicitação do seguro junto a esta IES;
    \item Entrega do Termo de Compromisso;
    \item Realização do ESO;
    \item Escrita do Relatório Técnico do ESO;
    \item Defesa do Relatório Técnico do ESO;
    \item Avaliação do Relatório Técnico do ESO;
    \item Entrega do Relatório Técnico do ESO.
\end{enumerate}

As explanações para as etapas apresentadas estão a seguir:

\begin{enumerate}
    \item \textbf{Matrícula na disciplina de ESO} – Somente poderão se matricular na disciplina de ESO os alunos que foram aprovados nas disciplinas de Banco de Dados, Redes de Computadores e Engenharia de Software. Esta decisão leva em conta que os estagiários devem estar com aproximadamente mais da metade da carga total do curso concretizada e já possuem um grau de conhecimento adequado para estagiar na área, tornando assim o estágio melhor desenvolvido e mais bem aproveitado para um futuro vínculo empregatício.
    \item \textbf{Solicitação do seguro junto a esta IES} – Segundo a Lei nº 11.788 a contratação de seguro de vida contra acidentes pessoais em favor do estagiário é obrigatória. Como o estágio é obrigatório para obtenção do diploma no curso de Bacharelado em Ciência da Computação, Unidade Acadêmica de Garanhuns da Universidade Federal Rural de Pernambuco, o seguro fica a cargo dessa instituição de ensino.
    \item \textbf{Entrega do Termo de Compromisso} – O Termo de Compromisso de Estágio é um acordo tripartite celebrado entre o educando, a parte concedente do estágio e a instituição de ensino, prevendo as condições de adequação do estágio à proposta pedagógica do  curso, à etapa e modalidade da formação escolar  do  estudante  e ao horário e calendário escolar. O termo deve ser entregue impresso em três vias, assinadas e carimbadas pela parte concedente, pelo  Supervisor,  pelo  Orientador, pelo Estagiário e pela Instituição.
    \item \textbf{Realização do ESO} – Em hipótese alguma o estágio pode ser iniciado sem a concretização das etapas 1, 2 e 3 apresentadas anteriormente. O estágio deve ser realizado sob supervisão de alguém formado na área de TI ou que possua no mínimo dois anos de experiência na área (comprovada via diploma ou declaração) e orientado por algum professor da UFRPE (dando preferência aos professores do curso de BCC/UAG). O estagiário não deverá ultrapassar 06 (seis) horas diárias e 30 (trinta) horas semanais para as atividades do estágio, assim sendo, é preciso estipular 10 semanas para concretização das 300 horas necessárias para o ESO em BCC, considerando que este tempo deve estar dentro do prazo para finalização do período corrente e da data limite para a defesa do relatório (dadas no calendário para cada semestre letivo).
    \item \textbf{Escrita do Relatório Técnico do ESO} – Após a realização das atividades do estágio e integralização da carga horária total o estagiário deve escrever o relatório técnico do estágio, apresentando as atividades realizadas, seguindo o modelo disponibilizado pelo curso. Somente após as correções sugeridas pelo professor orientador e o aval do mesmo para defesa, o estagiário deverá imprimir uma via (espiral), que deve ser entregue para a coordenação do curso, que por sua vez, repassa para a COE, responsável por marcar a data de defesa do relatório e sugerir as melhorias.
    \item \textbf{Defesa do Relatório Técnico do ESO} – O estagiário deverá realizar uma apresentação oral do Relatório de Estágio para o professor presidente da COE. A defesa visa avaliação e composição da nota final de ESO.
    \item \textbf{Avaliação do Relatório Técnico do ESO} – A composição da nota final do ESO (média na disciplina) será dada pela avaliação realizada pelo supervisor do estagiário na empresa, através de preenchimento de formulário padrão encaminhado pela Coordenação do Curso, conceito este responsável por 25\% da nota final; pela média das notas do professor-orientador do estagiário e do presidente da COE, estes dois últimos representam os 75\% restantes para composição da nota.
    \item \textbf{Entrega do Relatório Técnico do ESO} – O acadêmico deverá apresentar, após a correção final do relatório (sugeridas pelo presidente da COE após defesa), duas cópias em CD, para ficar na Coordenação do Curso e a outra que deverá compor o acervo da Biblioteca da Unidade, devendo seguir o padrão sugerido e adotado, para mais informações o aluno deve se dirigir à biblioteca da UAG.
\end{enumerate}

Como complemento, o Estágio não obrigatório, que é uma atividade com objetivo de proporcionar ao aluno a oportunidade de aplicar seus conhecimentos acadêmicos em situações de prática profissional pode ser realizado a partir do momento que o aluno é aprovado na disciplina de Algoritmos e Estrutura de Dados, assim como não pode ultrapassar as 120 (cento e vinte) horas para aproveitamento como Atividades Complementares.