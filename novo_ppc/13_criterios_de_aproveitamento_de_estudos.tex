\chapter{CRITÉRIOS DE APROVEITAMENTO DE ESTUDOS}

O aproveitamento de estudos corresponde à dispensa de cumprimento de disciplinas regulares do curso, quando a mesma ou uma equivalente em conteúdo e carga horária são cumpridas em outro curso superior, seja no âmbito da UFRPE ou de outra instituição.

Na UFRPE, a dispensa de disciplinas encontra-se normatizada pela Resolução CEPE/UFRPE nº 442/2006. Para que sejam creditadas, as disciplinas cursadas deverão:

\begin{enumerate}[label=(\alph*)]
  \item ser equivalentes em, pelo menos, 80\% (oitenta por cento) do conteúdo programático às correspondentes disciplinas que serão dispensadas; 	
  \item ter carga horária igual ou superior àquela das disciplinas a serem dispensadas; 	
  \item ser oferecidas regularmente pela Instituição onde foram cursadas como integrantes do currículo de um curso devidamente reconhecido.
\end{enumerate}

O pedido de dispensa da disciplina será dirigido ao coordenador do curso do solicitante, através de requerimento, acompanhado de histórico escolar ou declaração e do programa da disciplina a ser creditada. No requerimento deverão ficar esclarecidos códigos e denominações da disciplina a ser creditada e da disciplina a ser dispensada. Os pedidos de dispensa serão analisados por representantes dos cursos e homologados pelo CCD.

Este último terá a incumbência de realizar a dispensa das disciplinas não cursadas na UFRPE. Em se tratando de disciplina cursada na UFRPE, a dispensa será analisada e decidida diretamente pelo Coordenador, que informará ao CCD das dispensas, sendo obrigatório o registro em ata.

Existe a possibilidade de abreviação do tempo de formação para os alunos que demonstrem extraordinário aproveitamento nos estudos, como previsto na Lei nº 9.394/96, no Art. 47, § 2º. Este aparato legal ainda está em processo de regulamentação pela UFRPE com base na Resolução CFE nº 1/94 e no parecer CES/CNE n° 247/99.

