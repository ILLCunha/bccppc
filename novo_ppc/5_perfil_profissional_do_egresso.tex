\chapter{PERFIL PROFISSIONAL DO EGRESSO}

Deve se interessar pela computação e, em particular, pela ciência. Deve possuir entusiasmo para conhecer e dominar novos assuntos, além de disposição para construir sua própria reputação por meio dos produtos do esforço próprio ou resultantes de trabalho em equipe do qual participa. Deve possuir atitude e a necessidade de realizar, mesmo sem supervisão. Deve engajar-se em representações locais, regionais, nacionais e internacionais, através de representações de classe, visando a atualização e fortalecimento da sociedade. Estes atributos são esperados na conduta do estudante ingressante e utilizados ao longo do curso.

\section{Descrição dos requisitos psicofísicos}

Para atender ao perfil profissional definido, as atividades do curso priorizam o exercício dos requisitos inerentes ao desempenho da profissão, a citar:

\begin{itemize}
    \item Método e disciplina de trabalho;
    \item Raciocínio lógico e abstrato;
    \item Capacidade de trabalho em equipe;
    \item Criatividade, produtividade e iniciativa;
    \item Disposição para efetuar trabalho complexo e minucioso;
    \item Compromisso com o desenvolvimento tecnológico;
    \item Compromisso com o ser humano;
    \item Senso crítico, seriedade e responsabilidade.
\end{itemize}

\section{Egresso}

Do egresso de um curso de Bacharelado em Ciência da Computação é exigida uma predisposição e aptidões para a área, além de um conjunto de competências, habilidades e atitudes a serem adquiridas durante a realização do curso. Como perfil do Egresso, espera-se que ele ao final do curso, o profissional de Ciência da Computação necessita de flexibilidade para atender domínios diversificados de aplicação e as vocações institucionais. Dessa maneira, espera-se que os egressos desses cursos:

\begin{enumerate}
    \item Possuam sólida formação em Ciência da Computação e Matemática, que os capacitem a construir aplicativos de propósito geral, ferramentas e infraestrutura de software de sistemas de computação e de sistemas embarcados, gerar conhecimento científico e inovação, e que os incentivem a estender suas competências à medida que a área se desenvolve;
    \item Adquiram visão global e interdisciplinar de sistemas e entendam que esta visão transcende os detalhes de implementação dos vários componentes e os conhecimentos dos domínios de aplicação;
    \item Conheçam a estrutura dos sistemas de computação e os processos envolvidos na sua construção e análise;
    \item Dominem os fundamentos teóricos da área de Computação e como eles influenciam a prática profissional;
    \item Sejam capazes de agir de forma reflexiva na construção de sistemas de computação, compreendendo o seu impacto direto ou indireto sobre as pessoas e a sociedade;
    \item Sejam capazes de criar soluções, individualmente ou em equipe, para problemas complexos caracterizados por relações entre domínios de conhecimento e de aplicação.
\end{enumerate}

\section{Definição do perfil profissional}

Por definição, o Bacharel em Ciência da Computação deve ser um profissional qualificado para a pesquisa e desenvolvimento na área de computação, para o projeto e construção de hardware e software básico e também para o uso de sistemas computadorizados em outras áreas da atividade humana, a fim de viabilizar ou aumentar a produtividade e a qualidade de todos os tipos de procedimentos. Na UFRPE todo egresso deve ser um profissional: (1) com domínio e capacidade para trabalhar na área da Computação, desenvolvendo projetos de computadores e sistemas de computação, programas e sistemas de informação; (2) atento ao caráter ecológico, social e ético; e (3) que exerça suas atividades na sociedade com responsabilidade.

Adaptadas de documentos propostos pela ACM/IEEE e SBC, seguem as competências e habilidades necessárias para o egresso profissional de Ciência da Computação:

\begin{itemize}
    \item Possuir capacidade de raciocínio lógico e abstrato;
    \item Capacidade de utilizar conhecimentos de matemática, física, ciência da computação, engenharia e tecnologias modernas no apoio à construção de produtos e serviços seguros, confiáveis e de relevância social;
    \item Identificar práticas apropriadas dentro de um quadro ético, legal e profissional;
    \item Capacidade de atuar profissionalmente com ética avaliando o impacto de suas atividades no contexto social e ambiental;
    \item Reconhecer a necessidade de um desenvolvimento profissional contínuo;
    \item Capacidade para aprender a aprender. O aluno precisará estar sempre aprendendo para se manter atualizado e competente. A habilidade em pesquisa está fortemente relacionada com o auto-aprendizado;
    \item Discutir e explicar aplicações baseadas no corpo de conhecimento da computação;
    \item Visão sistêmica da área de computação;
    \item Profundo conhecimento dos aspectos teóricos, científicos e tecnológicos relacionados à área de computação;
    \item Demonstrar habilidade para trabalhar como um indivíduo sob orientação, como um membro de uma equipe ou como líder de uma equipe;
    \item Eficiência na operação de equipamentos computacionais e sistemas de software;
    \item Competência para identificar, analisar e documentar oportunidades, problemas e necessidades passíveis de solução via computação, e para empreender na concretização desta solução;
    \item Capacidade para pesquisar e viabilizar soluções de software para várias áreas de conhecimento e aplicação, como, por exemplo, desenvolvimento e/ou aprimoramento de protocolos de comunicação, modelos matemáticos-computacionais, técnicas de armazenamento de dados, construção de linguagens de programação, dentre inúmeras outras;
    \item Capacidade de abstração quando desenvolvendo as atividades de programação, projeto e modelagem;
    \item Compreender e aplicar conceitos, princípios e práticas essenciais no contexto de cenários bem definidos, mostrando discernimento na seleção e aplicação de técnicas e ferramentas;
    \item Compreensão da importância de se valorizar o  usuário  no  processo  de  interação com sistemas computacionais e competência na utilização de técnicas de interação homem-máquina neste processo;
    \item Conhecimento de aspectos relacionados à evolução da área de computação, de forma a poder compreender a situação presente e projetar a evolução futura;
    \item Capacidade para desenvolvimento de pesquisa científica e tecnológica, que permita ao aluno ingressar em um curso de pós-graduação ou realizar estas pesquisas na indústria;
    \item Capacidade de avaliar de forma aprofundada e com embasamento teórico as atividades realizadas e produtos desenvolvidos. Esta habilidade pode ser desenvolvida através de atividades de leitura e discussão de temas e elaboração de painéis de discussão com profissionais da área;
    \item Capacidade para conceber soluções inovadoras para tornar produtos competitivos;
    \item Capacidade de, com base nos conceitos adquiridos, iniciar, projetar, desenvolver, implementar, validar e gerenciar qualquer projeto de software. Este trabalho exige habilidade de solução de problemas e de avaliação crítica;
    \item Capacidade para projetar e desenvolver sistemas que integram hardware e software;
    \item Capacidade para avaliar prazos e custos em projetos de software;
    \item Competência e compromisso com a utilização de princípios e ferramentas que reduzam o tempo de desenvolvimento e implementação de um projeto e lhe confiram um alto grau de qualidade;
    \item Aplicação eficiente dos princípios de gerenciamento, organização e busca de informações;
    \item Conhecimento de aspectos relacionados às tecnologias de mídias digitais;
    \item Habilidade de lidar com notações, linguagens e ferramentas para elaboração de modelos;
    \item Capacidade empreendedora, inclusive para aqueles que não desejam ser empresários. Esta habilidade capacita o profissional a tomar iniciativas e a liderar projetos em suas atividades profissionais. Ela é desenvolvida nos alunos através de projetos nos quais eles são estimulados a apresentar e liderar projetos de sistemas;
    \item Capacidade de se expressar bem de forma oral ou escrita usando a língua portuguesa através da elaboração e apresentação de projetos e monografias durante todo o curso;
    \item Fluência na língua inglesa suficiente para a leitura e compreensão de documentos técnicos na área de computação.
    \item O aluno deve desenvolver competência e desempenho em língua inglesa através de disciplinas complementares e leitura de livros e artigos de computação escritos em Inglês, que são exigidos em várias atividades curriculares. 
\end{itemize}