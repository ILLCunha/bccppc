\chapter{TRABALHO DE CONCLUSÃO DE CURSO}

As disciplinas de Trabalho de Conclusão de Curso (TCC) são obrigatórias e fundamentais para desenvolvimento do discente, pois é uma oportunidade de consolidar e aprimorar os conhecimentos que adquiriu durante o curso. Tem por objetivo contribuir para a formação profissional, acadêmica e pessoal do aluno.

O Projeto do TCC complementa a formação do aluno, sendo seus resultados executados e descritos como pesquisa científica ou relatório técnico, compreendendo não somente trabalhos de pesquisa, mas trabalhos e serviços voltados para a indústria também.  Dessa maneira, a natureza do trabalho poderá ser como uma monografia, um artigo científico, um relatório técnico, entre outros. O tipo do projeto do TCC será definido entre o discente e seu orientador.

As disciplinas são ofertadas nos últimos períodos do curso e seus mecanismos de acompanhamento e de cumprimento são estabelecidos e acompanhados pelo professor responsável regidos pela resolução de TCC de BCC. A primeira disciplina possui 60 (sessenta) horas e tem o objetivo maior no entendimento do propósito, na construção e definição do projeto de TCC, onde através de aulas e acompanhamento do professor da disciplina, o discente, possa começar a executar seu trabalho e no semestre seguinte, matriculado na segunda disciplina, de 120 (cento e vinte) horas, com o projeto já adiantado, possa ter maiores chances de desenvolver seu trabalho no prazo esperado, diminuindo o risco de postergar seu curso em tanto tempo. Para cursar a disciplina de TCC 1 é necessário que o discente tenha integralizado pelo menos 50\% do curso.
