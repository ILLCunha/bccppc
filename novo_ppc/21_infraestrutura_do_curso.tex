\chapter{INFRAESTRUTURA DO CURSO}

O curso dispõe de basicamente toda a infraestrutura que a universidade oferece, como salas de aula, laboratórios de uso geral e de uso específico, biblioteca, restaurante/cantina, quadra poliesportiva, praças, entre outros. 

\section{Laboratórios}

Para a infraestrutura de informática, no que diz respeito aos laboratórios, a UAG oferece bons laboratório de software com conexão com a Internet (RNP) na qual os alunos possuem acesso no mínimo aos dois ambientes computacionais e de redes mais comuns atualmente: Windows e UNIX, como uma distribuição Linux, rodando em dual-boot. São 8 laboratórios de ensino com computadores e quadro com capacidade para 40 (quarenta) computadores, exclusivos para os cursos de computação.

A rede interna da UAG possui um backbone de 1 Gbps, cujo acesso à Internet é realizado através de um link de 100 Mbps fornecido pela Rede Nacional de Pesquisa (RNP). A UAG ainda disponibiliza, em boa parte do campus, acesso Wi-Fi para docentes, discentes e colaboradores, sendo as credenciais de acesso fornecidas pelo Sistema de Informações e Gestão Acadêmica (SIGA). Outro grande benefício da rede Wi-Fi da UAG é que ela faz parte da rede eduroam (education roaming), cujo principal benefício é a mobilidade. Ela permite o acesso sem fio à Internet localmente e em milhares de pontos de acesso no Brasil e no mundo com a utilização de uma mesma credencial.

Além disso, um laboratório de física com  instrumental necessário para matérias como arquitetura de computadores, circuitos digitais e automação: osciloscópios e analisadores digitais, kits de programação e simulação de sistemas de automação e de circuitos digitais. Destacamos que a proposta do Programa de Pós-Graduação em Ciência da Computação conta com 2 (dois) laboratórios exclusivos para os alunos da pós-graduação, como também conta com mais 2 (dois) laboratórios de uso compartilhado entre discentes de graduação e pós-graduação.

Num cenário de inovação e experimentação, o curso dispõe de duas iniciativas, a primeira diz respeito ao BCC Coworking\footnote{\url{http://bcc.uag.ufrpe.br/coworking}}, que é um Laboratório para Pesquisa e Desenvolvimento, com o propósito de servir como um local propício para o desenvolvimento de projetos reais, servindo até como fomento, com supervisão de profissionais da área, garantindo o conhecimento e a experiência técnica, através do uso de práticas e ferramentas do mercado de trabalho. 

O público alvo são os alunos que buscam a participação em projetos, assim como professores com projetos em execução (Pesquisa/Ensino/Extensão) com foco em aplicação comercial e/ou P\&D. Os tipos de atuação nesse laboratório podem ser diversos e as atividades podem ser formalizadas de maneiras diversas também, como por exemplo carga horária complementar em Pesquisa, Ensino, Extensão, na realização de ESO e TCC.

A segunda iniciativa no sentido de inovação é o Laboratório Multidisciplinar de Tecnologias Sociais - LMTS\footnote{\url{http://lmts.uag.ufrpe.br}}, que é um espaço permanente de ensino, pesquisa, inovação tecnológica, extensão e de colaboração com a gestão institucional, contando com colaboradores da área técnica, mas também das demais áreas citadas, sejam eles, professores, técnicos ou estudantes. Atualmente, é composto basicamente por discentes de graduação orientados por projetos de pesquisa e extensão, porém tem o desejo de receber alunos de pós-graduação que pudessem exercer a ciência com um nível de qualidade ainda maior. É um espaço que agrega inteligência coletiva e as múltiplas iniciativas em curso ou idealizadas em prol especificamente para o desenvolvimento de softwares livres ou públicos para atender as demandas da UFRPE e da sociedade em geral. Através desse laboratório é possível colocar em prática experimentos nos quesitos relacionados ao desenvolvimento de sistemas computacionais nas linhas de pesquisas propostas no projeto.

Além disso, temos ainda o laboratório temático do UNAME Research Group, composto por duas infraestruturas de sistemas distribuídos. A primeira se trata de um cluster composto por 6 computadores gerenciados pela plataforma Rocks Cluster, que é uma distribuição de cluster Linux de código aberto que permite que os usuários finais criem facilmente clusters computacionais. A segunda infraestrutura se trata de uma nuvem privada gerenciada pela plataforma OpenStack, que é um conjunto de projetos de software de código aberto usados para configurar e operar infraestrutura de computação e armazenamento em nuvem. Essa mesma infraestrutura de nuvem conta com um dispositivo de armazenamento de rede NAS, configurado em RAID, montado a partir de uma placa de expansão Orange Pi Win Plus e dois HDs de 1TB, cada. As duas infra-estruturas são usadas principalmente em experimentos de avaliação de desempenho, de planejamento de capacidade, de análise de disponibilidade e de investigação dos efeitos do envelhecimento de software. 

\section{Biblioteca}

A biblioteca da UFAPE tem sua origem em setembro de 2005, com a criação própria unidade, primeira extensão universitária instalada no País, dentro do programa de expansão do sistema federal de ensino superior. Pesquisar e registrar sua memória constitui importante tarefa para os que nela trabalham e representa um legado de episódios e contextos preservados para as gerações futuras.

O acervo da biblioteca recebe tratamento técnico (classificação, catalogação, indexação, tombamento, etc.) funcionando, de segunda a sexta-feira, das 08:00 às 21:00. Ocupa espaço físico equivalente a 08 salas de aulas mais dois corredores, distribuídos em 03 salas para o acervo geral; 01 sala para Literatura Cinzenta, Obras de Referência e Consulta Local; 01 sala para processamento técnico; 01 sala para Administração e Serviço de referência; 01 corredor para circulação de materiais com 02 guichês de atendimento; 02 totens de Consulta ao Acervo; 01 sala de estudos e 01 sala para o laboratório de informática (em processo de instalação).

Administrativamente, é subordinada à Direção Administrativa da UAG e tecnicamente ao SIBI-UFRPE. Seu acervo direciona-se a atender aos Cursos ofertados na UAG, tais como, Medicina Veterinária, Agronomia, Zootecnia, Pedagogia, Engenharia de Alimentos, Letras e Ciência da Computação. Oferece como serviços: empréstimo domiciliar; renovação e reservas on-line; catalogação na fonte; normalização; Comut; portal periódicos; multa solidária; visitas orientadas; aulas, palestras e cursos; terminais de consulta ao acervo e consulta local. 

O recursos disponíveis na biblioteca central (Prof. Mario Coelho de Andrade Lima) estão listados abaixo:

\begin{itemize}
    \item Número de Títulos do Acervo de Periódicos Impressos: 1.941
    \item Número de Títulos do Acervo de Livros Impressos: 48.227
    \item Número de Títulos de outros materiais: 1.247
    \item Número de Títulos do Acervo de Livro Eletrônico: 223.350
    \item Número de títulos na área de Educação: 8.610
    \item Possui Sinalização Tátil: Sim
    \item Possui Rampa de Acesso com corrimão: Sim
    \item Possui Entrada/Saída com Dimensionamento: Sim
    \item Possui Sinalização Sonora: Não
    \item Possui Sinalização Visual: Sim
    \item Possui Banheiros Adaptados: Sim
    \item Possui Espaço de Atendimento Adaptado: Sim
    \item Possui Mobiliário Adaptado: Sim
    \item Possui Acervo em Formato Especial (braile/sonoro): Sim
    \item Possui Sítios Desenvolvidos para Utilização dos Serviços Oferecidos: Sim
    \item Participa do Portal Capes de Periódicos: Sim
    \item Assina Bases de Dados: Sim
    \item Possui Repositório Institucional: Sim
    \item Utiliza Redes Sociais: Sim
    \item Oferece Serviços pela Internet: Sim
\end{itemize}

Já na biblioteca da Unidade Acadêmica, os recursos disponíveis estão listados abaixo:

\begin{itemize}
    \item Número de Títulos do Acervo de Livros Impressos: 4.235
    \item Número de Títulos de outros materiais: 238
    \item Número de Títulos do Acervo de Livro Eletrônico: 223.350
    \item Número de títulos na área de Educação: 2633
    \item Possui Rampa de Acesso com corrimão: Sim
    \item Possui Entrada/Saída com Dimensionamento: Sim
    \item Possui Sinalização Visual: Sim
    \item Possui Banheiros Adaptados: Sim
    \item Possui Mobiliário Adaptado: Não
    \item Possui Sítios Desenvolvidos para Utilização dos Serviços Oferecidos: Sim
    \item Participa do Portal Capes de Periódicos: Sim
    \item Assina Bases de Dados: Sim
    \item Possui Repositório Institucional: Sim
    \item Utiliza Redes Sociais: Sim
    \item Oferece Serviços pela Internet: Sim
\end{itemize}

Em outras bibliotecas, como na Unidade Acadêmica de Serra Talhada, no Colégio Dom Agostinho Ikas ou Unidade Acadêmica do Cabo de Santo Agostinho, os recursos são listados abaixo:

\begin{itemize}
    \item Acervo total geral das Bibliotecas: 11.902
    \item Empréstimo Domiciliar
    \item Empréstimo Especial
    \item Empréstimo Interbibliotecas
    \item Reservas on-line
    \item Renovações on-line
    \item Catalogação na Fonte
    \item Normalização
    \item COMUT
    \item Reservas de espaços
    \item BDTD da UFRPE\footnote{\url{www.bdtd.ufrpe.br}}
    \item Portal Periódicos
    \item Portal de Periódicos da UFRPE\footnote{\url{www.journals.ufrpe.br}}
    \item Visitas Orientadas
    \item Núcleo do Conhecimento Prof. João Baptista (memória UFRPE)
    \item Coleção de Livros Eletrônicos Ebrary
    \item Treinamentos e cursos
\end{itemize}

Acrescenta-se ainda que a Unidade Acadêmica de Garanhuns está recebendo um novo prédio para Biblioteca e salas de estudos, um prédio maior e bem completo, projetado exclusivamente para ser uma Biblioteca, com previsão de conclusão para 2020.
