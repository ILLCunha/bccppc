\chapter{OBJETIVOS DO CURSO}

A Ciência da Computação é uma área de atuação bem diversa, uma vez que o profissional pode atuar em diferentes segmentos, inclusive, apenas na aplicação da computação como meio para outras áreas do conhecimento. O curso de BCC tem como objetivo formar profissionais com bases científicas e tecnológicas na área da computação, capazes de resolver problemas dos mais diferentes domínios, através de métodos e técnicas computacionais, para atuar de forma bem sucedida tanto na área acadêmica quanto no mercado de trabalho. 

Os objetivos específicos do curso são:

\begin{enumerate}
    \item Desenvolver nos estudantes o perfil científico de pesquisador, tanto para atuação na área acadêmica, quanto para atuação em outros ramos de atividade;
    \item Desenvolver nos estudantes um espírito empreendedor, incentivando e motivando a sua independência e criatividade;
    \item Desenvolver nos discentes o perfil para trabalhar na indústria, aplicando os seus conhecimentos técnicos de desenvolvimento de software e soluções para TI;
    \item Promover a interdisciplinaridade buscando atualização constante na área de computação;
    \item Motivar e orientar o estudante para que ele tenha uma postura ativa diante da necessidade de um aprendizado contínuo e autônomo;
    \item Promover uma postura ética e socialmente comprometida com o papel do estudante no desenvolvimento científico, tecnológico, social e econômico da sua região e do País;
    \item Promover interação constante com escolas do ensino fundamental e médio local, de forma a estimular vocações e colaborar ativamente com a melhoria da educação.
    
\end{enumerate}