\chapter*{APRESENTAÇÃO}
\addcontentsline{toc}{chapter}{Apresentação}

Nos dias atuais, o reconhecimento do direito à educação em termos de acesso, permanência e qualidade se faz presente na sociedade brasileira, constituindo em uma agenda de alta prioridade. Neste sentido, as Instituições Públicas de Ensino Superior, fortalecidas pelas políticas afirmativas e inclusivas, vêm contribuindo de maneira expressiva para o desenvolvimento socioeconômico, cultural e tecnológico do país, nas mais variadas áreas do conhecimento humano. É diante dessa conjuntura que a \acrfull{ufape} / \acrfull{ufrpe} reafirma seu compromisso com o desenvolvimento de uma sociedade crítica e participativa através da construção e popularização de saberes científicos, tecnológicos e culturais (UFRPE, 2018).

Atento às demandas sociais, econômicas e culturais do Estado de Pernambuco e, em especial, da Região do Agreste do referido estado, este Projeto Pedagógico se apresenta como um instrumento político e teórico-metodológico destinado a pautar as práticas acadêmicas do Curso de Graduação em Ciência da Computação ofertado pela UFAPE/UFRPE através da Unidade Acadêmica de Garanhuns (UAG), criada pela Resolução CONSU/UFRPE nº 98/2017.

O curso tem o compromisso de contribuir para a transformação social sustentável, formando profissionais que possam atuar de forma ética e inovadora, respeitando os aspectos legais e as normas inerentes à profissão. Sua concepção está em consonância com a Lei de Diretrizes e Bases da Educação, as Diretrizes Curriculares Nacionais para a área em questão, bem como o Plano de Desenvolvimento Institucional (PDI) da UFRPE e os dispositivos legais da Universidade e Entidades de Classe, como a Sociedade Brasileira de Computação. Além disso, sua proposta pedagógica orienta-se por uma \textit{concepção ativa dos processos de ensino e aprendizagem}, incorporando metodologias que incentivam a criatividade e autonomia dos estudantes. Neste sentido, destaca-se o ensino híbrido e a realização de \textit{projetos interdisciplinares} por meio da \textit{Problem Based Learning} (PBL) em diferentes etapas do curso.

Tal como elucida Veiga (2004), o Projeto Pedagógico não representa um documento estanque ou um ``artefato'' meramente técnico. Pelo contrário, devido a sua dinamicidade, ele atua como um mobilizador permanente para todos os agentes envolvidos com o curso: professores, estudantes, técnico-administrativos e gestores. A fim de garantir a formação, este Projeto deverá ser permanentemente acompanhado e avaliado, com vistas à realização do seu aperfeiçoamento e à efetivação da sua intencionalidade.
