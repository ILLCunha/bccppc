\chapter{ATIVIDADES COMPLEMENTARES}

As atividades complementares têm a finalidade de propiciar saberes e habilidades que enriqueçam o processo de ensino e aprendizagem, possibilitando a ampliação dos conhecimentos didáticos, curriculares, científicos e culturais por meio de atividades realizadas nos mais diversos espaços (Unidades Acadêmicas da Universidade, ONGs, Instituições públicas e privadas, etc). Essas atividades de formação complementar abrangem as modalidades de ensino, pesquisa e extensão, bem como as suas formas de registro no histórico escolar, devidamente detalhadas na Resolução CEPE/UFRPE nº 362/2011. 

Ainda de acordo com a resolução supracitada, em seu Art. 1º, Parágrafo único, “toda atividade acadêmica complementar deverá ficar sobre a responsabilidade, de, pelo menos, um professor, devendo ser avaliada e homologada pelo Colegiado de Coordenação Didática – CCD do curso”. Neste sentido, o acompanhamento e avaliação dessas atividades estarão integrados ao planejamento do curso. O aluno deverá, obrigatoriamente, apresentar uma ou mais atividades de naturezas distintas, sejam Ensino, Pesquisa ou Extensão. O Quadro abaixo apresenta uma breve amostra de atividades complementares previstas para o Bacharelado em Ciência da Computação.

\begin{center}
  
  \begin{scriptsize}
    \begin{longtable}{p{2.5cm}p{3.5cm}p{6cm}p{2cm}}
      \caption{\label{quadro:atividades-complementares}Atividades complementares previstas para o curso.}\\
      \toprule
      \textbf{Modalidade} & \textbf{Atividade} & \textbf{Descrição} & \textbf{C.H}\\ 
        \midrule
        Formação \newline Profissional & Estágio não Obrigatório & Atividade que tem o objetivo de proporcionar ao aluno a oportunidade de aplicar seus conhecimentos acadêmicos em situações de prática profissional. & Não exceder 120 horas\\
        \addlinespace
        & Cursos de Formação Profissional Complementar & Cursos ofertados à comunidade sob a forma de Educação Continuada, objetivando a socialização do conhecimento acadêmico, potencializando o processo de interação universidade-sociedade. & \\
        & Pesquisa de Iniciação Científica & Conjunto de atividades ligadas a programas e projetos de pesquisa desenvolvidos pelo Aluno, sob orientação do Docente. & \\
        \addlinespace
        & Realização de Visita técnica & Visitas a lugares de interesse para a área de formação que complementem o conteúdo das disciplinas, relacionando teoria e prática. & \\
        \midrule
        Extensão Universitária e Aperfeiçoamento & Projetos de Extensão & Ações processuais, de caráter educativo, cultural, artístico, científico e/ou tecnológico, que envolvem Docentes, Alunos e Técnico-administrativos, e que são desenvolvidas junto à comunidade, mediante ações sistematizadas. & Não exceder 120 horas \\
        \addlinespace
        & Participação em Eventos de Extensão (internos e externos) & Participação em Congressos, Seminários, Jornadas e similares, que possuam o propósito de produzir, sistematizar, divulgar e intercambiar conhecimentos, tecnologias e bens culturais. & \\
        \addlinespace
        & Apresentação de Trabalhos em Eventos & Apresentação oral de trabalhos acadêmicos em Congressos, Seminários, Jornadas e similares. & \\ \addlinespace
        & Publicação científica & Divulgação dos resultados da investigação através da produção de artigos. & \\ \addlinespace
        & Prestação de serviços à comunidade & Participação em atividades que possibilitem a transferência à comunidade do conhecimento gerado no âmbito do curso. & \\ \midrule
        Experiência de Ensino & Monitoria & Ação de cooperação dos corpos discente e docente nas atividades de ensino, pesquisa e extensão efetuadas em trabalhos de laboratório, biblioteca, de campo e outras compatíveis com seu nível de conhecimento e experiência nas disciplinas e desenvolver habilidades que favoreçam o Aluno na iniciação à docência. & Não exceder 120 horas \\ \addlinespace
        & Participação em Projetos de Ensino & Participação, em ações devidamente reconhecidas pela Universidade de acordo com a legislação vigente, que tem por objetivo estimular e apoiar as ações de Ensino, não curriculares, com caráter temporário, complementar e/ou de aprofundamento, que visam à melhoria do processo de ensino aprendizagem dos cursos de Graduação ofertados na UFAPE e que tenham como público alvo membros internos da comunidade universitária. São exemplos de projetos de ensino: cursos, grupos de estudo e eventos. & \\ \addlinespace
        Políticas & Representação discente em comissões e comitês
        Participação em órgãos colegiados da UFRPE. & Participação em órgãos colegiados da UFRPE.& \\ \addlinespace
        Empreendedorismo e Inovação & Participação em Empresas Júnior, incubadoras ou outros mecanismos & Participação, desenvolvimento e execução de projetos. & \\ \addlinespace
        & Desenvolvimento de protótipo ou produto & Produção de materiais. & \\
      \bottomrule
    \end{longtable}
    \nota[Fonte]{Adaptado dos Referenciais da SBC, 2017 e da Resolução CEPE/UFRPE nº 362/2011.}
  \end{scriptsize}      
\end{center}

A carga horária total das atividades complementares para o curso de Ciência da Computação é de 320h. Esta será considerada apenas mediante o requerimento protocolado à Coordenação do Curso e acompanhado da documentação comprobatória.
